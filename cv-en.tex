\documentclass[11pt,a4paper]{moderncv} 
\usepackage[utf8]{inputenc}  
\usepackage[scale=0.75]{geometry}
\recomputelengths

\renewcommand{\familydefault}{\sfdefault}

\moderncvstyle{classic}                             % style options are 'casual' (default), 'classic', 'banking', 'oldstyle' and 'fancy'
\moderncvcolor{blue}                               % color options 'black', 'blue' (default), 'burgundy', 'green', 'grey', 'orange', 'purple' and 'red'

\name{Danila}{Usachev}
\title{Curriculum Vitae}
\email{duesna897@gmail.com}
\social[github]{sharkov63}
\extrainfo{
    Telegram: \href{https://t.me/usachevd0}{usachevd0} \\
    Codeforces: \href{https://codeforces.com/profile/usachevd0}{usachevd0}
}

\begin{document}
\maketitle

\section{My Stack}

\cvlistitem{C/C++ programming (mostly standard C++17 and under)}
\cvlistitem{Compiler internals as in LLVM \& Clang codebase}
\cvlistitem{Lua, Python}
\cvlistitem{Kotlin, Java}

\section{Professional Experience}

\subsection{Software Engineering}

\cventry{2022--now}{Software Engineer, Researcher}{Huawei R\&D, Saint-Petersburg Research Center, Network Software Engineering and Verification Laboratory}{Saint-Petersburg, Russia}{}{Developed and maintained a static analysis tool written in C++ and based on Clang, to analyze, test and verify network-related codebases in C language. \\ Researched approaches to optimize pipelines of testing and verification of network code.}

\subsection{Academy}

\cventry{2022}{Member of Innopolis Open 2022 Scientific Committee}{Innopolis University}{Innopolis, Russia}{}{Problem developer for the aforecited competition in informatics.}
\cventry{2021}{Teacher}{Summer School for Olympiad Training}{Innopolis University}{Innopolis, Russia}{}

\section{Competitive Programming}
Starting from 2017 (7th year at school) I've had a long and successful experience in competitive programming, with multiple national and international achievements. See my \href{https://codeforces.com/profile/usachevd0}{\color{blue} \textit{Codeforces page}}. My best achievements are:

\cvlistitem{\textbf{Silver medal}, the 33rd International Olympiad in Informatics (IOI), 2021.}
\cvlistitem{\textbf{Gold medal}, the 8th Romanian Master of Informatics (RMI), 2020.}
\cvlistitem{\textbf{Gold medal}, the 12th International Autumn Tournament of Informatics (IATI), 2020.}
\cvlistitem{ \textbf{Winner} of the 33rd Russian National
Olympiad in Informatics, 2021, Moscow, Russia.}

\section{Education}

\cventry{2021--2025}{B.Sc.~programme ``Modern Software Engineering''}{St. Petersburg State University, Department of Mathematics and Computer Science}{}{}{At the moment (2023-2024) I am a third-year Bachelor student.}

\section{Research Interests}

\cvlistitem{Static and dynamic code analysis}
\cvlistitem{Symbolic execution}
\cvlistitem{Mathematical Logic}
\section{Pet-projects}

\cvlistitem{\textbf{SAKLS}: \emph{Syntax-Aware Keyboard Layout Switching} project for automatically switching layout in a text editor. Available as \href{https://github.com/sharkov63/sakls.nvim}{\color{blue}sakls.nvim} Neovim plugin and \href{https://github.com/sharkov63/sakls}{\color{blue}sakls} backend library.}

\cvlistitem{\textbf{Utilities for speedrunning} GoldSRC and Source engine video games (such as Half-Life): \href{https://github.com/sharkov63/SaveWarpTool}{\color{blue} \texttt{SaveWarpTool}} (C++) and \href{https://github.com/sharkov63/SaveWarpFinder}{\color{blue} \texttt{SaveWarpFinder}} (Python).}

\cvlistitem{\textbf{Intellij IDEA plugin for finding ``code owner''}: given version control system data, it finds the contributor which has the best knowledge of a given file (\href{https://github.com/sharkov63/code-owner-finder}{\color{blue} \texttt{code-owner-finder}}). Written in Kotlin.}

\cvlistitem{\textbf{Compiler} of a fictional language: a test task in VK internship selection for KPHP team (\href{https://github.com/sharkov63/vk-test-cpmc}{\color{blue} \texttt{vk-test-cpmc}}). Written in C++.}

\cvlistitem{Kotlin projects during ``Programming Fundamentals'' course in the first semester of undergraduate. The best one is \textbf{visual diagram building} program (\href{https://github.com/sharkov63/pf-2021-viz}{\color{blue} \texttt{pf-2021-viz}}), others can also be found on GitHub.}


\section{Other Activities}

\subsection{Amateur Game Developing in Unity (JavaScript, C\#)}

\cventry{2014-2016}{RcBjVh \normalfont(\href{https://drive.google.com/file/d/0B2kALEIFlKmjaV9ZWExBelpqM28/view?resourcekey=0-NZv8IgRi0ZbLzLHVM-oqMA}{\textit{\color{blue}link}})}{}{}{}{}

\cventry{2015}{GrowJump \normalfont(\href{https://www.dropbox.com/s/r1o5nnbtgwa5mcb/frowJump.rar?dl=0}{\textit{\color{blue}link}})}{as part of Ludum Dare 34 competition}{}{}{}

\indent

\subsection{Speedrunning}

At school I used to do \textit{speedrunning} --- attempting to complete certain video games as fast as possible. Aside from real-time speedrunning, I was the head of the following projects.

\indent

\cventry{2016-2018}{Half-Life: Done Enormously Warped \normalfont(\href{https://www.youtube.com/watch?v=J0n-TIrpkDA}{\color{blue} \textit{link}})}{}{}{}{A segmented speedrun of Half-Life, based on a game-breaking glitch, which allows teleporting in large distances by transitioning between maps. Utilities \href{https://github.com/sharkov63/SaveWarpTool}{\color{blue} \texttt{SaveWarpTool}} and \href{https://github.com/sharkov63/SaveWarpFinder}{\color{blue} \texttt{SaveWarpFinder}} mentioned above were developed exactly for this project. }

\cventry{2019-2020}{Tool-assisted speedrun of I Wanna Be The Boshy \normalfont(\href{https://tasvideos.org/6843S}{\color{blue} \textit{link}})}{}{}{}{A speedrun that was created using third-party tools with the goal of achieving ``theoretically perfect speedrun''.}

\end{document}
