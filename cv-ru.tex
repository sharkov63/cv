\documentclass[11pt,a4paper]{moderncv} 
\usepackage[utf8]{inputenc}  
\usepackage[russian]{babel}
\usepackage[scale=0.75]{geometry}
\recomputelengths

\renewcommand{\familydefault}{\sfdefault}

\moderncvstyle{classic}                             % style options are 'casual' (default), 'classic', 'banking', 'oldstyle' and 'fancy'
\moderncvcolor{blue}                               % color options 'black', 'blue' (default), 'burgundy', 'green', 'grey', 'orange', 'purple' and 'red'

\name{Данила}{Усачев}
\title{Curriculum Vitae}
\email{duesna897@gmail.com}
\social[github]{sharkov63}
\extrainfo{
    Telegram: \href{https://t.me/usachevd0}{usachevd0} \\
    Codeforces: \href{https://codeforces.com/profile/usachevd0}{usachevd0}
}

\begin{document}
\maketitle

\section{Стэк}

\cvlistitem{Разработка на C/C++ (в основном стандарт C++17 и ниже)}
\cvlistitem{Внутренности компиляторов: LLVM и Clang}
\cvlistitem{Lua, Python}
\cvlistitem{Kotlin, Java}

\section{Опыт работы}

\subsection{Разработка ПО}

\cventry{2022--\ldots}{Разработчик-исследователь}{Huawei R\&D, Санкт-Петербургский исследовательский центр, лаборатория <<разработки и верификации сетевого ПО>>}{Санкт-Петербург, Россия}{}{Разрабатывал и поддерживал инструмент статического анализа кода, написанный на C++ и основанный на Clang, предназначенный для анализа, тестирования и верификации сетевого ПО на языке C. \\ Исследовал подходы оптимизации пайплайнов тестирования и верификации сетевого кода.}

\subsection{Академия}

\cventry{2022}{Член научного комитета олимпиады Innopolis Open 2022}{Университет Иннополис}{Иннополис, Россия}{}{Разрабатывал задачи для вышеуказанного соревнования по информатике.}
\cventry{2021}{Преподаватель}{Летняя школа олимпиадной подготовки}{Университет Иннополис}{Иннополис, Россия}{}

\section{Спортивное программирование}
Начиная с 2017 (7 класс в школе) у меня был длинный и успешный опыт в спортивном программировании, с несколькими всероссийскими и международными победами. См. мой \href{https://codeforces.com/profile/usachevd0}{\color{blue}профиль на Codeforces}. Мои ключевые достижения:

\bigskip

\cvlistitem{\textbf{Серебряная медаль}, 33-я международная олимпиада по информатике (IOI), 2021.}
\cvlistitem{\textbf{Золотая медаль}, 8-я международная олимпиада RMI, 2020.}
\cvlistitem{\textbf{Золотая медаль}, 12-я международная олимпиада IATI, 2020.}
\cvlistitem{\textbf{Призёр} 33-й всероссийской олимпиады школьников по информатике, 2021, Москва, Россия.}

\newpage
\section{Образование}

\cventry{2021--2025}{Бакалавриат <<Современное программирование>>}{Санкт-Петербургский государственный университет, факультет математики и компьютерных наук}{}{}{В настоящее время (2023-2024) я учусь на 3 курсе бакалавриата}

\section{Исследовательский интерес}

\cvlistitem{Статический и динамический анализ кода}
\cvlistitem{Символьное исполнение}
\cvlistitem{Математическая логика}

\section{Пет-проекты}

\cvlistitem{\textbf{SAKLS} (\emph{Syntax-Aware Keyboard Layout Switching}) --- проект для автоматического переключения раскладки клавиатуры в текстовом редакторе, в зависимости от синтаксиса. Доступен как плагин \href{https://github.com/sharkov63/sakls.nvim}{\color{blue}sakls.nvim} для редактора Neovim, и как библиотека-бэкенд \href{https://github.com/sharkov63/sakls}{\color{blue}sakls}.}

\cvlistitem{\textbf{Утилиты для скоростного прохождения игр} на игровых движках GoldSRC и Source (например, Half-Life): \href{https://github.com/sharkov63/SaveWarpTool}{\color{blue} \texttt{SaveWarpTool}} (C++) и \href{https://github.com/sharkov63/SaveWarpFinder}{\color{blue} \texttt{SaveWarpFinder}} (Python).}

\cvlistitem{\textbf{Плагин Intellij IDEA для нахождения ``code owner''}: по данным системы контроля версий найти человека, у которого больше всего знаний о данном файле (\href{https://github.com/sharkov63/code-owner-finder}{\color{blue} \texttt{code-owner-finder}}). Написано на Kotlin.}

\cvlistitem{\textbf{Компилятор} игрушечного языка: тестовое задание для стажировки VK в команду KPHP (\href{https://github.com/sharkov63/vk-test-cpmc}{\color{blue} \texttt{vk-test-cpmc}}). Написан на C++.}

\cvlistitem{Программа для построения графиков \href{https://github.com/sharkov63/pf-2021-viz}{\color{blue} \texttt{pf-2021-viz}}, написанная на Kotlin.}

\section{Прочее}

\subsection{Любительская разработка игр в Unity (JavaScript, C\#)}

\cventry{2014-2016}{RcBjVh \normalfont(\href{https://drive.google.com/file/d/0B2kALEIFlKmjaV9ZWExBelpqM28/view?resourcekey=0-NZv8IgRi0ZbLzLHVM-oqMA}{\textit{\color{blue}link}})}{}{}{}{}

\cventry{2015}{GrowJump \normalfont(\href{https://www.dropbox.com/s/r1o5nnbtgwa5mcb/frowJump.rar?dl=0}{\textit{\color{blue}link}})}{в рамках соревнования Ludum Dare 34}{}{}{}

\indent

\subsection{Скоростное прохождение игр}

Во время школы я занимался <<спидраннингом>> --- попытками пройти некоторые компьютерные игры как можно быстрее. Помимо скоростных прохождений в реальном времени, я возглавлял следующие проекты.

\indent

\cventry{2016-2018}{Half-Life: Done Enormously Warped \normalfont(\href{https://www.youtube.com/watch?v=J0n-TIrpkDA}{\color{blue} \textit{link}})}{}{}{}{Сегментированное скоростное прохождение игры Half-Life, основанное на критическом глюке, позволяющем телепортиваться на огромные дистанции за счёт переходов между картами. Утилиты \href{https://github.com/sharkov63/SaveWarpTool}{\color{blue} \texttt{SaveWarpTool}} и \href{https://github.com/sharkov63/SaveWarpFinder}{\color{blue} \texttt{SaveWarpFinder}}, упомянутые выше, были разработаны именно для этого проекта.}

\cventry{2019-2020}{Tool-assisted speedrun of I Wanna Be The Boshy \normalfont(\href{https://tasvideos.org/6843S}{\color{blue} \textit{link}})}{}{}{}{Скоростное прохождение, созданное с помощью сторонних утилит, с целью получить <<теоретически идеальное прохождение>>.}

\end{document}
